\documentclass[a4paper, 11 pt]{article}

\usepackage[latin1]{inputenc}
\usepackage[italian]{babel	}
\usepackage{indentfirst}
\usepackage{graphicx}
\usepackage{subfigure}
\usepackage{mathtools}
\usepackage[retainorgcmds]{IEEEtrantools}
\usepackage{multirow}
\usepackage{array}
\usepackage{mhchem}
%\usepackage{psfrag}
\usepackage[hypcap=true]{caption}
\usepackage[bookmarks=true,hyperfootnotes=false]{hyperref}
\hypersetup{
	colorlinks=true,
	linkcolor=red,
	anchorcolor=red,
	citecolor=red,
	urlcolor=red,
	%linktocpage=true
	pdftitle={THGEM study},
	pdfauthor={I. Ciraldo, G.A. Brischetto}
}

\title{\bf {\huge THGEM study} }
\author{Dott. D. Torresi, I. Ciraldo, G.A. Brischetto}
\date{16 Dicembre 2019}

\begin{document}

\maketitle

\section{Introduction}

Due parole sull'obiettivo dei test, caratterizzazione delle THGEM e del prototipo, focalizzandoci sulle risposte del sistema variando le THGEM

Descrivere i due tipi di THGEM: foto, schemi, differenze, tabella FULL THGEM vs. ROW THGEM (indicando pitch, dimensione del foro e rim, spessore totale e spessore della metallizzazione (copper plating) )



\section{Apparato di misura}
parliamo di CAEN e PICO

diagramma sui canali della lettura della corrente

inserire il plot run190 in cui si vede la misura fatta con PICO e quella con CAEN (per tutti e sei i canali!!)

larghezza delle thgem � 107per107 mm2 e ci sono 143 file di buche-> 20,000 buche

gas isobutano

flow 145 sccm (standard cubic centimeter per minute)


range asimmetrico del bottom1, che ha minore precisione degli altri canali (farsi dire i valori da Alfonso)

\section{Method}

sorgente alpha e shutter, aprire e chiudere shutter, variazione di corrente

caratteristiche sorgente (55 kBq, 241Am)

rate di 140 Hz di particelle alpha

\section{Misure}

i canali 3 e 4 possono essere ignorati perch� non danno mai segnale

spieghiamo i diversi scan (induzione, thgem e drift)


\subsection{FULL THGEM}

partiamo dalle full thgem:

sulla parte di induzione: uno dei grafici con i quattro canali e la somma appropriata, 
incrocio fra corrente anodo e top3, regione di quasi plateau, poi la corrente aumenta, se avremo delle simulazione potremo corredare il discorso; obiettivo: cercare la regione di migliore funzionamento che dovrebbe essere quella del plateau



sulla parte delle thgem: andamento esponenziale, si raggiunge il limite di scarica, non sappiamo se la scarica � nelle thgem



sulla parte di drift: variare la corrente tra catodo e bottom1; a 0 volt abbiamo una misura: o lo strumento segna zero ma non � zerp 



descrizione delle varie pressioni: un grafico tenendo soltanto anodo a diverse pressioni, questo grafico non varia drammaticamente al variare della pressione; un altro grafico con l'induzione al variare della tensione delle thgem e/o del drift.



ion backflow: calcolare direttamente il valore in percentuale; unico grafico al variare delle pressioni; esso non dipende molto dall'induzione; esso dipende anche delle thgem (configurazione delle linee di forza); conclusione: 

fattore di moltiplicazione: inizialmente solo con le thgem piene;
poi thgem row e mostrare i tre grafici esemplari
induzione delle row thgem a 20 mbar � pi� bello;

\subsection{ROW THGEM}

la drift � da discutere: la loro geometria � diversa: 5 fila isolate e proviamo a spiegare che a basse tensioni della drift, il campo elettrico delle thgem forma un imbuto pi� largo, se vdrift aumenta la regione da cui le thgem raccolgono carica diminuisce

dalle row thgem ci aspettiamo meno corrente perch� hanno meno buchi: invece di 143 file, ne hanno 5, quindi hanno il 3.5\% dei buchi rispetto a quelle piene. In prima approssimazione potremmo aspettarci che il valore della corrente scali allo stesso modo. 

ion backflow nelle row � sistematicamente pi� alto; vediamo se l'andamento � simile; al variare dell'induzione diminuzione dell'IBF; 

%ricordarsi di fare la misura del drift delle row a 20 mbar
%ATTENZIONE: altri test tenendo i campi elettrici fissi
%Misura ad alta pressione con full thgem, misura ad altre pressione con row thgem e rifare i test cercando di mantenere i campi elettrici costanti (possibilmente con il drift a tensioni basse)



\section{Conclusioni}




\end{document}